\chapter{Introduction}\label{ch:introduction}

\section{Motivation}\label{sec:motivation}
    \subsection{Driving simulation}\label{subsec:driving-simulation}
        The use of driving simulators is common practice among researchers for studying driving behaviors in scenarios that may include dangerous and unlikely situations\cite{Wynne2019,That2011}.
        % TODO: reformulate
        Utilization of driving simulators is primarily driven by the significant reduction in time and financial resources for engineers and developers, as evidenced by reference\cite{That2011}.
        Additionally, it facilitates expeditious research and evaluation of innovative human-machine-interface designs (HMI) within the automotive industry\cite{riegler2023}.
        % "me"-perspective zittieren?
        One such simulator represents the "me"-perspective by focusing the simulation on the vehicle that is driven by a human in the simulator, thereby ensuring that this vehicle is modeled with great precision.
        One illustrative example of a well-known driving simulator is CARLA, which is an open-source, free-to-use driving simulator with a feature-rich Python API\cite{carla2017}.

    \subsection{Microscopic traffic simulation}\label{subsec:microscopic-traffic-simulation}
        The use of microscopic traffic simulators is also a common methodology among researchers studying the dynamics of vehicular traffic, particularly in the context of intelligent transportation systems (ITS)\cite{toledo2005microscopic}.
        One such microscopic traffic simulator employs (among other models like: lane-change, fuel-consumption, \ldots) car-following models to regulate the acceleration and, consequently, the movement of each vehicle\cite{treiber2013traffic}.
        A variety of car-following models are available to use for modelling different types of vehicles.
        One notable example is the IDM (Intelligent Driver Model) (see Section~\ref{subsec:intelligent-driver-model}) described by Springer et al., which assembles a complete and accident free models, that produces plausible acceleration profiles\cite{treiber2013traffic}.
        % TODO:Richtig Zittiert??
        Moreover, as described in~\cite{treiber2013traffic} these models are derived from a set of assumptions about real drivers such as keeping a "safe distance" from the leading vehicle, driving at a desired speed or preferring acceleration to be within a comfortable range, etc.
        As all the mentioned Parameters are like sensor-values to a modern ACC (adaptive cruse control) System with the model being the control system.
        Since these models are fully deterministic, they don`t model the "errors" of human drivers.

    \subsection{Integrated simulation environment}\label{subsec:integrated-simulation-environment}

        \begin{figure}
            \centering
            \begin{tikzpicture}[scale=5]
                \node[name=p, shape=pilot, minimum size=1.5cm](P1){Participant};
                \node[module, right= 2cm of P1] (I1) {driving simulator};
                \node[module, right= 1cm of I1] (I2) {traffic simulator};
                \draw[<->] (I1)--(I2);
                \draw[->] (I1)--(P1.mouth);
                \draw[->, ] (P1.east)--(I1);
            \end{tikzpicture}
            \caption{Integration architecture}
            \label{fig:int:architecture}
        \end{figure}

        The goal of investigating the impact of human drivers on traffic flow can be achieved by modeling those imperfections of a human driver.
        Using so called meta-models on top of  typical vehicle-following models which modify the behavior to closely match a real human driver.
        Example for meta-models are the human driver model (HDC)\cite{treiber2006delays} or the extended human driver model (EHDC)\cite{Lindorfer2018}.
        An additional potential methodology for the observation of human behavior is to engage directly in a scenario, which offers the benefit of eliminating the necessity for a model of the human driver.
        As a result of the direct participation of the human in the scenario, the impact of that behavior on the overall traffic flow can be observed.
        The aforementioned "me"-perspective is transformed into a holistic "we"-perspective through the utilization of an integrated simulation environment~\cite{riegler2023}.
        This kind of integration has proven very useful\cite{punzo2010integration}.

        There are two types of integrations:
        \begin{itemize}
            \item \textbf{Offline integration:} For this integration a driving simulator and a traffic simulator are used seperatly.
            For example one could build the same scenario in both simulators and a human driver could participate in the scenario inside the driving simulator.
            Afterward the behavior of the human driver is recorded and imported into the traffic simulator, so the impact of the recorded behavior can be observed.
            The key disadvantages of this integration are, that it lacks the possibility of observing the impact of the driving behavior in real time and the model-controlled vehicles are not able to adapt to the human driver's behavior during the simulation.
            \item \textbf{Online integration:} The aforementioned disadvantages are rendered obsolete by implementation of an online integration.
            The most straightforward structure for such an integration would be to have the participant interact with the driving simulator, which is then synchronized to the traffic simulator.
            This can be achieved by connecting a driving simulator with a microscopic traffic simulator, as illustrated in Figure\ref{fig:int:architecture}.
            By running the simulators in sync, it is possible for the simulation controlled (by the traffic simulator) vehicles to react to the driving actions of the vehicle controlled by a human driver.
        \end{itemize}
        The remaining work focuses on the online integration of the simulators, unless otherwise specified.

\section{Objective of the work}\label{sec:objective-of-the-work}
    In this thesis, an integrated Simulation Environment is proposed, consisting of the driving simulator CARLA\cite{carla2017} and the microscopic traffic simulator TraffSim\cite{backfrieder2013traffsim,backfrieder2014traffsim}.
    The primary objective of this integration is to achieve a robust and synchronized environment, where the real-time interactions between human-driven and vehicular-follow-model-controlled vehicles unfold seamlessly.

    The key challenges of the proposed integrated environment are:
    \begin{itemize}
        \item \textbf{Synchronisation of integration components:} {
            One of the primary objectives of this work is to ensure seamless synchronisation between the driving and traffic simulators.
            This synchronisation involves real-time data exchange, particularly for online integration, where the actions of a human driver in the driving simulator should immediately impact vehicle behaviours in the traffic simulator.
            A significant challenge in this regard is the minimisation of latency, which could cause inconsistent or delayed responses between simulators.
        }
        \item \textbf{Road network compatibility and flexibility:}{
            Synchronizing road networks across the simulators is essential for a coherent simulation environment, as inconsistencies can lead to inaccuracies in traffic flow or navigation.
            Therefore, a procedure for exporting, importing and exchaning road networks between the simulators was implemented by the use of the OpenDrive\ref{openDrive} Standard.
        }
        \item \textbf{Handling human driving behavior:}{
            Human drivers often exhibit unpredictable behaviors, creating a need for adaptable models that can respond effectively to erratic or non-standard actions.
            This objective was partially archived by extrapolating the acceleration and speed of human-driven vehicle, for further use, this objective has to be readressed.
            One potential solution for this problem could be to use a vehicle-following, which models a human driver\cite{Lindorfer2018}, to predict the next action taken by the human in charge of the vehicle.
        }
    \end{itemize}

\section{Structure}\label{sec:structure}
    This thesis is structured into multiple chapters, each addressing discrete aspects of the integration of driving and traffic simulators and the associated challenges.\\

    The initial chapter, entitled "Introduction", provides an overview of the subject, outlining the historical development of driving and traffic simulation, the individual functionalities of each, and the rationale for integrating them into a unified environment.
    Furthermore, it delineates the objectives and scope of the work.

    The second chapter, entitled "Basics", identifies the foundational studies of the field of traffic research, that are related to the discussed topic.

    The third chapter, entitled "Related Work", presents a review of existing research on driving and traffic simulation systems.
    It focuses on previously explored methods for integration, identifying both successful techniques and common obstacles encountered.

    The fourth chapter, entitled "Methods", provides a detailed account of the architectural design of the integrated simulation environment.
    It describes the technical framework and design decisions that have been taken in order to achieve synchronisation of the simulators, to facilitate real-time data exchange and to ensure compatibility between simulator platforms.

    The fifth chapter, Results, presents the outcomes of the implemented integration, including a discussion of performance metrics, an analysis of the observed behaviours of the simulation components, and an evaluation of the solutions proposed for specific integration challenges.

    The sixth chapter, entitled "Future Work", identifies areas for further research and development.
    It mainly discusses future use cases for the proposed integrated environment in addition to a description of the necessary steps to transform the already working proof of concept into a usable tool for traffic research.